//Пролин част в тм доменах
//Если убрать - будет плохо
//Ключевая роль

//Геометрия! Кинки - Proline-induced distortions of transmembrane helices.
//Вспывает наверх (!) - Conserved positioning of proline residues in membrane-spanning helices of ion-channel proteins
---> может индуцировать ТМ-ТМ: Proline localized to the interaction interface can mediate self-association of transmembrane domains

//Возможность изомеризации - посмотреть в A model for transmembrane helix with a cis-proline in the middle. (Обсуждение, что таки возможно)

Принято считать, что остаток пролина в трансмембранных спиралях играет функциональную роль. На это указывает сравнительно высокая встречаемость пролина в трансмембранных спиралях по сравнению со спиралями водорастворимых белков \cite{prolineExp1}, \cite{bacteriorhodopsin}. Также известно, что внесение мутаций в Pro в ТМ домене транспортных белков приводит к деактивации их механизма действия \cite{prolineExp2}, что тоже говорит о ключевой роли пролина в активности трансмембранных доменов. \\

Как компьютерное моделирование, так и экспериментальные данные показывают, что пролин вносит конформационные изменения в геометрию ТМ $\alpha$-спирали \cite{model}, \cite{prolineDistortions}. Цис- конформация пептидной связи между XXX-Pro создает так называемый "кинк" в $\alpha$-спирали, что может изменять ее ориентацию в мембране и, как следствие, менять свои функциональные свойства. Также было замечено, что пролин в трансмембранных спиралях с цис- конформацией имеет достаточно консервативное положение: он чаще всего оказывается "вытащен" в область гидрофильных головок \cite{positioning}. 